\documentclass{article}

% these packages let you do math
\usepackage{amsmath}
\usepackage{amssymb}

% we need these packages for fancy R tables
\usepackage{booktabs}
\usepackage{float}
\usepackage{colortbl}
\usepackage{xcolor}

% these packages play with the spacing/margins of the document. Uncomment the commands on lines 16 and 17 to see what they do.
\usepackage{a4wide}
\usepackage{setspace}
\usepackage{geometry}
\usepackage{parskip}
%\doublespacing
%\geometry{margin=1.5in}

% this package helps us with including images. Setting the graphics path makes it easier to refer to things in the \include graphics command.
\usepackage{graphicx}
\graphicspath{ {../figures/} }

% make some hyperlinks using the \href command
\usepackage{hyperref}
\hypersetup{
    colorlinks=true,
    linkcolor=black,
    urlcolor=blue
}

% set the author, title, and date of the document. \maketitle adds it to the document.
\author{Blake Lin}
\title{Incarceration Rates by Race and Gender in the U.S.}
\date{February 18, 2022}

\begin{document}
\maketitle

In this report, we are using the data from National Longitudinal Survey of Youth to analyze the Incarceration status by race and gender. 



\begin{figure}[H]
    \begin{center}
        \includegraphics[width=.85\textwidth]{incarc_by_racegender}
    \end{center}
     \caption{Mean Months in Incarceration in 2002 by Race and Gender}
    \label{fig:graph}
\end{figure}

In figure 1, we can see the average month under incarceration of different race and gender. Black male has the highest number of approximately 0.5 which means they are incarcerated for almost 5 months on average. We can also see that Male has higher numbers than females across all races. 
(Note: the problem of zero mean in mixed race male might be the result of small cells.)

\input{../tables/incarc_by_racegender.tex}
In table 1, we can see the actual numbers of the mean months in incarceration.


\input{../tables/regress_incarc_by_racegender.tex}
In table 2, we have the regression output, and the omitted category is "Black Females".

Being hispanic, mixed race, or non-black / non-hispanic would both decrease the months being in incarceration, and a male would increase the expected months in incarceration for almost 2 months on average. All of the variables we used are statistically significant under a 95\% level.







\end{document}